% Copyright (C) 1997, 2002--2020 Free Software Foundation, Inc.

% Author: Stephen Gildea <stepheng+emacs@gildea.com>
% Russian translation: Alex Ott <alexott@gmail.com>

% This document is free software: you can redistribute it and/or modify
% it under the terms of the GNU General Public License as published by
% the Free Software Foundation, either version 3 of the License, or
% (at your option) any later version.

% As a special additional permission, you may distribute reference cards
% printed, or formatted for printing, with the notice "Released under
% the terms of the GNU General Public License version 3 or later"
% instead of the usual distributed-under-the-GNU-GPL notice, and without
% a copy of the GPL itself.

% This document is distributed in the hope that it will be useful,
% but WITHOUT ANY WARRANTY; without even the implied warranty of
% MERCHANTABILITY or FITNESS FOR A PARTICULAR PURPOSE.  See the
% GNU General Public License for more details.

% You should have received a copy of the GNU General Public License
% along with GNU Emacs.  If not, see <https://www.gnu.org/licenses/>.

\documentclass[10pt]{article}
\usepackage{multicol,tabularx}
\usepackage[a4paper,hmargin={2cm,2cm},vmargin={2cm,2cm},nohead,twoside]{geometry}
\usepackage[T2A]{fontenc}
\usepackage[utf8]{inputenc}
\usepackage[english,russian]{babel}
\usepackage{verbatim}
%\usepackage{enumerate,calc}
%\usepackage{indentfirst,amsmath,graphicx,citehack,floatflt}
%
%\usepackage{pscyr}
%\renewcommand{\rmdefault}{fha}

\newlength{\ColWidth}
\setlength{\ColWidth}{120mm}
\newlength{\ColThreeWidth}
\setlength{\ColThreeWidth}{25mm}

\newcommand{\versionemacs}[0]{28} % version of Emacs this is for
\newcommand{\cyear}[0]{2020}       % copyright year

\newcommand\shortcopyrightnotice[0]{\vskip 1ex plus 2 fill
  \centerline{\footnotesize \copyright\ \cyear\ Free Software Foundation, Inc.
  Permissions on back.}}

\newcommand\copyrightnotice[0]{
\vskip 1ex plus 2 fill\begingroup\footnotesize
\centerline{Copyright \copyright\ \cyear\ Free Software Foundation, Inc.}
\centerline{For GNU Emacs version \versionemacs}
\centerline{Designed by Stephen Gildea}
\centerline{Перевод Alex Ott <alexott@gmail.com>}

\centerline{Released under the terms of the GNU General Public License
  version 3 or later.}

\centerline{For more Emacs documentation, and the \TeX{} source for this card,}
\centerline{see the Emacs distribution, or {\tt https://www.gnu.org/software/emacs}}
\endgroup}

\hyphenation{mini-buf-fer}

\parindent 0pt
%\parskip 0.5ex plus .2ex minus .2ex

\newcommand{\kbd}[1]{\texttt{#1}}

\begin{document}

%\begin{multicols}{3}

\centerline{Справочник команд GNU Emacs}
\centerline{(для версии \versionemacs)}

\section{Запуск Emacs}

Для запуска GNU Emacs \versionemacs, просто наберите имя программы: \kbd{emacs}

\section{Выход из Emacs}

\begin{tabular}{p{\ColWidth}p{\ColThreeWidth}p{\ColThreeWidth}}
Приостановить работу Emacs (или свернуть в иконку при работе в X) & \kbd{C-z} \\
Выйти из Emacs & \kbd{C-x C-c} \\
\end{tabular}

\section{Файлы}

\begin{tabular}{p{\ColWidth}p{\ColThreeWidth}p{\ColThreeWidth}}
{\bf открыть} файл в Emacs & \kbd{C-x C-f} \\
{\bf сохранить} файл обратно на диск & \kbd{C-x C-s} \\
сохранить {\bf все} файлы & \kbd{C-x s} \\
{\bf вставить} содержимое другого файла в текущий буфер & \kbd{C-x i} \\
открыть нужный вам файл вместо текущего & \kbd{C-x C-v} \\
сохранить буфер в указанном файле & \kbd{C-x C-w} \\
Переключить состояние буфера ``только для чтения'' & \kbd{C-x C-q} \\
\end{tabular}

\section{Получение помощи}

Справочная система достаточно проста. Hажмите \kbd{C-h} (или \kbd{F1}) и
следуйте инструкциям.  Если вы в первый раз работаете с Emacs, то наберите
\kbd{C-u C-h t Russian RET} для вызова {\bf учебника}.

\begin{tabular}{p{\ColWidth}p{\ColThreeWidth}p{\ColThreeWidth}}
Скрыть окно справки & \kbd{C-x 1} \\
Прокрутить окно справки & \kbd{C-M-v} \\

apropos: показать команды, соответствующие строке & \kbd{C-h a} \\
показать имя функции, которая запускается клавишей & \kbd{C-h k} \\
получить информацию о функции & \kbd{C-h f} \\
получить информацию о режиме & \kbd{C-h m} \\
\end{tabular}

\section{Исправление ошибок}

\begin{tabular}{p{\ColWidth}p{\ColThreeWidth}p{\ColThreeWidth}}
{\bf прервать} частично набранную или исполняемую команду & \kbd{C-g} \\
{\bf восстановить} файл, потерянный при крахе системы & \kbd{M-x recover-file} \\
{\bf отменить} нежелательные изменения & \kbd{C-x u, C-\_ {\rm или} C-/} \\
восстановить буфер в первоначальное состояние & \kbd{M-x revert-buffer} \\
перерисовать заполненный мусором экран & \kbd{C-l} \\
\end{tabular}

\section{Hаращиваемый поиск}

\begin{tabular}{p{\ColWidth}p{\ColThreeWidth}p{\ColThreeWidth}}
поиск вперед & \kbd{C-s} \\
поиск назад & \kbd{C-r} \\
поиск по регулярному выражению вперед & \kbd{C-M-s} \\
поиск по регулярному выражению назад & \kbd{C-M-r} \\

выбрать предыдущую строку поиска & \kbd{M-p} \\
выбрать следующую строку поиска & \kbd{M-n} \\
прекратить наращиваемый поиск & \kbd{RET} \\
отменить эффект ввода последнего символа & \kbd{DEL} \\
прервать текущий поиск & \kbd{C-g} \\
\end{tabular}

Для повтора поиска в любом направлении используйте клавиши \kbd{C-s} или
\kbd{C-r}. Если Emacs все еще производит поиск, \kbd{C-g} отменит только
не выполненную часть.

\section{Перемещение}

\begin{tabular}{p{\ColWidth}p{\ColThreeWidth}p{\ColThreeWidth}}
{\bf единицы перемещения} & {\bf назад} & {\bf вперед} \\
символ & \kbd{C-b} & \kbd{C-f} \\
слово & \kbd{M-b} & \kbd{M-f} \\
строка & \kbd{C-p} & \kbd{C-n} \\
начало (или конец) строки & \kbd{C-a} & \kbd{C-e} \\
предложение & \kbd{M-a} & \kbd{M-e} \\
абзац & \kbd{M-\{} & \kbd{M-\}} \\
страница & \kbd{C-x [} & \kbd{C-x ]} \\
выражение (sexp) & \kbd{C-M-b} & \kbd{C-M-f} \\
функция & \kbd{C-M-a} & \kbd{C-M-e} \\
переход в начало (или конец) буфера & \kbd{M-<} & \kbd{M->} \\

\end{tabular}

\begin{tabular}{p{\ColWidth}l}
прокрутка экрана вниз & \kbd{C-v} \\
прокрутка экрана вверх & \kbd{M-v} \\
прокрутка влево & \kbd{C-x <} \\
прокрутка вправо & \kbd{C-x >} \\
установка текущей строки в центр экрана & \kbd{C-u C-l} \\
\end{tabular}

\section{Уничтожение и удаление}

\begin{tabular}{p{\ColWidth}p{\ColThreeWidth}p{\ColThreeWidth}}
{\bf уничтожаемый объект} & {\bf назад} & {\bf вперед} \\
символ (удаление, а не уничтожение) & \kbd{DEL} & \kbd{C-d} \\
слово & \kbd{M-DEL} & \kbd{M-d} \\
строка (до конца строки) & \kbd{M-0 C-k} & \kbd{C-k} \\
предложение & \kbd{C-x DEL} & \kbd{M-k} \\
выражение & \kbd{M-- C-M-k} & \kbd{C-M-k} \\
\end{tabular}


\begin{tabular}{p{\ColWidth}l}
уничтожить {\bf область} & \kbd{C-w} \\
скопировать область в список уничтожений & \kbd{M-w} \\
уничтожить до следующего вхождения символа {\it char} & \kbd{M-z {\it char}} \\

вставить последний уничтоженный объект & \kbd{C-y} \\
заменить только что восстановленный текст предшествующим уничтоженным текстом & M-y \\
\end{tabular}

\section{Пометка}

\begin{tabular}{p{\ColWidth}l}
установить пометки & \kbd{C-@ {\rm или} C-SPC} \\
поменять местами курсор и пометку & \kbd{C-x C-x} \\

пометить {\it arg\/} {\bf слов} & \kbd{M-@} \\
пометить {\bf параграф} & \kbd{M-h} \\
пометить {\bf страницу} & \kbd{C-x C-p} \\
пометить {\bf выражение} & \kbd{C-M-@} \\
пометить {\bf функцию} & \kbd{C-M-h} \\
пометить весь {\bf буфер} & \kbd{C-x h} \\
\end{tabular}

\section{Замена с запросом}

\begin{tabular}{p{\ColWidth}l}
интерактивная замена текстовой строки & \kbd{M-\%} \\
с использованием регулярных выражений & \kbd{M-x query-replace-regexp} \\
\end{tabular}

В режиме интерактивной замены пользователю доступны следующие команды:

\begin{tabular}{p{\ColWidth}l}
{\bf заменить} данное соответствие, перейти к следующему & \kbd{SPC} \\
заменить это соответствие, не перемещаться & \kbd{,} \\
{\bf пропустить} соответствие без замены & \kbd{DEL} \\
заменить все оставшиеся соответствия & \kbd{!} \\
{\bf вернуться} к предыдущему соответствию & \kbd{\^} \\
{\bf выйти} из режима ``замена с запросом'' & \kbd{RET} \\
войти в режим рекурсивного редактирования (для выхода используется \kbd{C-M-c}) & \kbd{C-r} \\
\end{tabular}

\section{Использование нескольких окон}

Команды, приведенные во втором столбце, применяются к другому
фрэйму (окну используемой оконной системы).

\begin{tabular}{p{\ColWidth}l}
удалить все остальные окна & \kbd{C-x 1} \\
\end{tabular}

\begin{tabular}{p{\ColWidth}p{\ColThreeWidth}p{\ColThreeWidth}}
разбить окно по горизонтали & \kbd{C-x 2\ \ \ \ } & \kbd{C-x 5 2} \\
удалить данное окно & \kbd{C-x 0\ \ \ \ } & \kbd{C-x 5 0} \\

разбить окно по вертикали & \kbd{C-x 3} \\

выполнить прокрутку в другом окне & \kbd{C-M-v} \\
\end{tabular}

\begin{tabular}{p{\ColWidth}p{\ColThreeWidth}p{\ColThreeWidth}}
переместить курсор в другое окно & \kbd{C-x o} & \kbd{C-x 5 o} \\

выбрать буфер в другом окне & \kbd{C-x 4 b} & \kbd{C-x 5 b} \\
показать буфер в другом окне & \kbd{C-x 4 C-o} & \kbd{C-x 5 C-o} \\
загрузить файл в другое окно & \kbd{C-x 4 f} & \kbd{C-x 5 f} \\
открыть файл в режиме чтения в другом окне & \kbd{C-x 4 r} & \kbd{C-x 5 r} \\
запустить Dired в другом окне & \kbd{C-x 4 d} & \kbd{C-x 5 d} \\
найти таг в другом окне & \kbd{C-x 4 .} & \kbd{C-x 5 .} \\

\end{tabular}

\begin{tabular}{p{\ColWidth}l}
увеличить высоту окна & \kbd{C-x \^} \\
сузить текущее окно & \kbd{C-x \{} \\
расширить текущее окно & \kbd{C-x \}} \\
\end{tabular}

\section{Форматирование}

\begin{tabular}{p{\ColWidth}l}
сделать отступ для текущей {\bf строки} (в зависимости от режима) & \kbd{TAB} \\
сделать отступ для {\bf области} (в зависимости от режима) & \kbd{C-M-$\backslash$} \\
сделать отступ для {\bf выражения} (в зависимости от режима) & \kbd{C-M-q} \\
сделать отступ к заданной {\it arg\/} колонке & \kbd{C-x TAB} \\

вставить новую строку после курсора & \kbd{C-o} \\
сдвинуть остаток строки вертикально вниз & \kbd{C-M-o} \\
удалить пустые строки вокруг курсора & \kbd{C-x C-o} \\
объединить строку с предыдущей (со следующими arg) & \kbd{M-\^} \\
удалить пустое пространство вокруг курсора & \kbd{M-$\backslash$} \\
поместить только один пробел в позиции курсора & \kbd{M-SPC} \\

отформатировать текущий параграф & \kbd{M-q} \\
установить столбец заполнения & \kbd{C-x f} \\
установить префикс заполнения каждой строки & \kbd{C-x .} \\

установить начертание (face) & \kbd{M-g} \\
\end{tabular}

\section{Изменение регистра}

\begin{tabular}{p{\ColWidth}l}
сделать буквы слова заглавными & \kbd{M-u} \\
сделать буквы слова строчными & \kbd{M-l} \\
сделать первую букву слова заглавной & \kbd{M-c} \\

сделать буквы области заглавными & \kbd{C-x C-u} \\
сделать буквы области строчными & \kbd{C-x C-l} \\
\end{tabular}

\section{Минибуфер}

Следующие клавиши (ключи) определены в минибуфере.

\begin{tabular}{p{\ColWidth}l}
дополнить насколько возможно & \kbd{TAB} \\
дополнить до одного слова & \kbd{SPC} \\
дополнить и выполнить & \kbd{RET} \\
показать возможные дополнения & \kbd{?} \\
выбрать предыдущую строку из истории ввода & \kbd{M-p} \\
выбрать следующую строку из истории ввода & \kbd{M-n} \\
поиск в истории по регулярному выражению в обратном напрвлении & \kbd{M-r} \\
поиск в истории по регулярному выражению в прямом напрвлении & \kbd{M-s} \\
прервать команду & \kbd{C-g} \\
\end{tabular}

Наберите \kbd{C-x ESC ESC} для редактирования и повторения последней
команды набранной в минибуфере. Наберите \kbd{F10}, чтобы работать с
пунктами меню при помощи команд минибуфера.

\section{Буфера}

\begin{tabular}{p{\ColWidth}l}
выбрать другой буфер & \kbd{C-x b} \\
выдать список всех буферов & \kbd{C-x C-b} \\
уничтожить буфер & \kbd{C-x k} \\
\end{tabular}

\section{Перестановка данных}

\begin{tabular}{p{\ColWidth}l}
обменять местами {\bf символы} & \kbd{C-t} \\
обменять местами  {\bf слова} & \kbd{M-t} \\
обменять местами  {\bf строки} & \kbd{C-x C-t} \\
обменять местами  {\bf выражения} & \kbd{C-M-t} \\
\end{tabular}

\section{Проверка правописания}

\begin{tabular}{p{\ColWidth}l}
проверить текущее слово & \kbd{M-\$} \\
проверить все слова в области & \kbd{M-x ispell-region} \\
выполнить проверку всего буфера & \kbd{M-x ispell-buffer} \\
\end{tabular}

\section{Теги}

\begin{tabular}{p{\ColWidth}l}
найти определение тега & \kbd{M-.} \\
найти следующее вхождение тега & \kbd{C-u M-.} \\
использовать новый файл с тегами & \kbd{M-x visit-tags-table} \\

поиск по шаблону по всей таблице тегов & \kbd{M-x tags-search} \\
выполнить query-replace над всеми файлами & \kbd{M-x tags-query-replace} \\
продолжить поиск или поиск-замену тега & \kbd{M-,} \\
\end{tabular}

\section{Командные процессоры}

\begin{tabular}{p{\ColWidth}l}
Выполнить команду командного процессора & \kbd{M-!} \\
выполнить команду командного процессора над областью & \kbd{M-|} \\
пропустить содержимое области через команду командного процессора & \kbd{C-u M-|} \\
запустить командный процессор в окне \kbd{*shell*} & \kbd{M-x shell} \\
\end{tabular}

\section{Прямоугольные области}

\begin{tabular}{p{\ColWidth}l}
скопировать прямоугольную область в регистр & \kbd{C-x r r} \\
удалить прямоугольную область & \kbd{C-x r k} \\
вставить последную уничтоженную прямоугольную область & \kbd{C-x r y} \\
открыть прямоугольную область, сдвигая текст вправо & \kbd{C-x r o} \\
очистить прямоугольную область & \kbd{C-x r c} \\
предварить каждую строку строкой текста & \kbd{C-x r t} \\
\end{tabular}

\section{Сокращения}

\begin{tabular}{p{\ColWidth}l}
добавить глобальное сокращение & \kbd{C-x a g} \\
добавить локальное для режима сокращение & \kbd{C-x a l} \\
добавить глобальное значение для данного сокращения & \kbd{C-x a i g} \\
добавить специфическое для режима значение данного сокращения & \kbd{C-x a i l} \\
явно вставить значение для сокращения & \kbd{C-x a e} \\

динамически дополнить предыдущее слово & \kbd{M-/} \\
\end{tabular}

\section{Регулярные выражения}

\begin{tabular}{p{\ColWidth}l}
любой одиночный символ (за исключением символа новой строки) & \kbd{. {\rm(dot)}} \\
ноль или более повторений & \kbd{*} \\
одно или более повторений & \kbd{+} \\
ноль или одно повторение & \kbd{?} \\
экранировать специальный символ {\it c\/} & \kbd{$\backslash${\it c}} \\
выбор между альтернативами (``или'') & \kbd{$\backslash|$} \\
группировка & \kbd{$\backslash$( {\rm$\ldots$} $\backslash$)} \\
текст в {\it n\/}-й группе & \kbd{$\backslash${\it n}} \\
граница слова & \kbd{$\backslash$b} \\
не граница слова & \kbd{$\backslash$B} \\
\end{tabular}

\begin{tabular}{p{\ColWidth}p{\ColThreeWidth}p{\ColThreeWidth}}
{\bf объект} & {\bf начало соответствия} & {\bf конец соответствия} \\
строка & \kbd{\^} & \kbd{\$} \\
слово & \kbd{$\backslash$<} & \kbd{$\backslash$>} \\
буфер & \kbd{$\backslash$`} & \kbd{$\backslash$'} \\
\end{tabular}

\begin{tabular}{p{\ColWidth}p{\ColThreeWidth}p{\ColThreeWidth}}
{\bf класс символов} & {\bf соответ\-ствует им} & {\bf соответ\-ствует другим} \\
явный набор & \kbd{[ {\rm$\ldots$} ]} & \kbd{[\^ {\rm$\ldots$} ]} \\
символ-буква & \kbd{$\backslash$w} & \kbd{$\backslash$W} \\
символ с синтаксисом {\it c} & \kbd{$\backslash$s{\it c}} & \kbd{$\backslash$S{\it c}} \\
\end{tabular}

\section{Международные наборы символов}

\begin{tabular}{p{\ColWidth}l}
выбрать язык & \kbd{C-x RET l} \\
показать все возможные методы ввода & \kbd{M-x list-input-methods} \\
разрешить или запретить метод ввода & \kbd{C-$\backslash$} \\
установить систему кодирования для следующей команды & \kbd{C-x RET c} \\
показать список всех систем кодирования & \kbd{M-x list-coding-systems} \\
выбрать предпочтительную систему кодирования & \kbd{M-x prefer-coding-system} \\
\end{tabular}

\section{Информация (система Info)}

\begin{tabular}{p{\ColWidth}l}
запустить систему доступа к документации & \kbd{C-h i} \\
найти указанную функцию или переменную в документации & \kbd{C-h S} \\
\end{tabular}

Перемещение внутри узла (нодами):

\begin{tabular}{p{\ColWidth}l}
прокрутка вперед & \kbd{SPC} \\
прокрутка назад & \kbd{DEL} \\
перейти к началу узла & \kbd{. {\rm (dot)}} \\
\end{tabular}

Перемещение между узлами:

\begin{tabular}{p{\ColWidth}l}
{\bf следующий} узел & \kbd{n} \\
{\bf предыдущий} узел & \kbd{p} \\
перейти к {\bf верхнему} узлу & \kbd{u} \\
выбрать пункт меню по имени & \kbd{m} \\
выбрать {\it n\/}-й пункт меню по номеру (1--9) & \kbd{{\it n}} \\
перейти по ссылке (возврат с помощью \kbd{l}) & \kbd{f} \\
возвратится к последнему просмотренному узлу & \kbd{l} \\
возвратится к узлу-каталогу & \kbd{d} \\
Перейти к верхему узлу файла & \kbd{t} \\
перейти к узлу, заданному именем & \kbd{g} \\
\end{tabular}

Другие команды:

\begin{tabular}{p{\ColWidth}l}
запустить {\bf учебник} по системе документации & \kbd{h} \\
поиск темы в индексе & \kbd{i} \\
поиск в узлах с помощью регулярных выражений & \kbd{s} \\
{\bf прекратить работу} с документацией & \kbd{q} \\
\end{tabular}

\section{Регистры}

\begin{tabular}{p{\ColWidth}l}
сохранить область в регистре & \kbd{C-x r s} \\
вставить содержимое регистра в буфер & \kbd{C-x r i} \\

сохранить положение курсора в регистре & \kbd{C-x r SPC} \\
перейти к положению курсора, сохраненному в регистре & \kbd{C-x r j} \\
\end{tabular}

\section{Клавиатурные макросы}

\begin{tabular}{p{\ColWidth}l}
{\bf начать} определение клавиатурного макроса & \kbd{C-x (} \\
{\bf закончить} определение клавиатурного макроса & \kbd{C-x )} \\
{\bf выполнить} последний определенный макрос & \kbd{C-x e} \\
дополнить последний определенный макрос & \kbd{C-u C-x (} \\
задать имя для последнего определенного макроса & \kbd{M-x name-last-kbd-macro} \\
вставить в буфер определение макроса на языке Lisp & \kbd{M-x insert-kbd-macro} \\
\end{tabular}

\section{Команды, связанные с Emacs Lisp}

\begin{tabular}{p{\ColWidth}l}
вычислить {\bf выражение} перед курсором & \kbd{C-x C-e} \\
вычислить текущую функцию {\bf defun} & \kbd{C-M-x} \\
вычислить {\bf область} как выражение & \kbd{M-x eval-region} \\
считать и вычислить выражение из минибуфера & \kbd{M-:} \\
загрузить из стандартного системного каталога & \kbd{M-x load-library} \\
\end{tabular}

\section{Простая настройка}

\begin{tabular}{p{\ColWidth}l}
настроить переменные и оформление (faces) & \kbd{M-x customize} \\
\end{tabular}

% The intended audience here is the person who wants to make simple
% customizations and knows Lisp syntax.

Создание глобальных привязок клавиш (ключей) на Emacs Lisp (примеры):

\begin{verbatim}
(global-set-key "\C-cg" 'goto-line)
(global-set-key "\M-#" 'query-replace-regexp)
\end{verbatim}

\section{Написание команд}

\begin{verbatim}
(defun <command-name> (<args>)
  "<documentation>" (interactive "<template>")
  <body>)
\end{verbatim}

Пример:

\begin{verbatim}
(defun this-line-to-top-of-window (line)
  "Reposition line point is on to top of window.
With ARG, put point on line ARG."
  (interactive "P")
  (recenter (if (null line)
                0
              (prefix-numeric-value line))))
\end{verbatim}

Спецификация \kbd{interactive} сообщает о том, что аргумент надо считывать
интерактивно. Для дополнительной информации наберите \kbd{C-h f interactive}.

\copyrightnotice

%\end{multicols}

\end{document}

% Local variables:
% compile-command: "pdflatex ru-refcard"
% coding: utf-8
% End:
